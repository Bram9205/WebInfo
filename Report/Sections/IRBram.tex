\subsection{Indexing of notifications}
\textbf{Name:} Bram Kohl \indent \textbf{StudentNumber:} 0746107

\subsubsection*{Motivation}
When the P2000 data has been retrieved, it is just raw data. There is no clear structure in it yet and as such, it has to be indexed, so we can easily use it later on. For example, in order to be able to associate the P2000 notifications with the messages from the news feed and twitter data, we want to be able to easily and quickly obtain information about the notifications. Information like the date and time the emergency service was called, which emergency service was called (fire department, police department or ambulance), with what urgence the emergency service was called, where the incident happened and suchlike. So what we want to do is extract a lot of information from the raw (html) data and possibly from some other sources (e.g. get name of town from postal code). We want to store that information in a structured and easily accessible way.
\subsubsection*{Approach}
The indexing part kicks in when the raw data has already been retrieved. This data contains html code which has a structure we can work with. It consists of a table with a few rows per notification. Using the rows and columns of this table we can extract the date and time the emergency service was called, the type of emergency service it concerns and the region the incident happened (25 possible regions, so about 2 regions per province). Furthermore, if there are capcodes present (codes which indicate the region and type of emergency service) we extract those as well.\\

Furthermore, we extract a table field which contains a message for the notification. This message often contains a postal code and/or the name of a town. As we would like to know the location more specific than just the region, we try to extract this information from the message.
The postal code can be extracted using the structure of a postal code (4 numbers and two letters, with an optional space in between). We do this using a regular expression.\\

The name of a town can be obtained in two ways. If we find a postal code in the message, we obtain it using the api at \url{http://www.postcodeapi.nu}. This is the most reliable way of the two. If we don't have a postal code, we attempt to extract it from the P2000 message using a database of towns in the Netherlands. We created this database from a list of 5712 towns that we found online (\url{http://home.kpn.nl/pagklein/almanak.html}). We check for each town whether it occurs in the message as a whole word and if it does, we store the town. If we detect multiple towns, we store all of them so the association analyzer can take all of them into account.\\

When we have isolated and transformed all the data we need, we store it in SQL database tables with a clear structure, so we can later easily access the data. We create indexes on this database on several fields (e.g. date and time and postal code) to speed up future search queries.
\subsubsection*{Evaluation}
The indexing of notifications will be evaluated by manually checking the resulting data in the database, to see whether the data has been split up properly, and by running queries and testing how long they take to see whether the indexes are working properly. These manual tests showed that (in the end) everything works as expected.