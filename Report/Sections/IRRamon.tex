\subsection{Document P2000 notification codes \& Collect Twitter data}
\textbf{Name:} Ramon de Vaan \indent \textbf{StudentNumber:} 0758873

\subsubsection*{Motivation}
Most of the P2000 notifications contain an extra attribute, which provides some extra data regarding the nature of the incident.
This attribute comes in the form of a notification code.
We need to translate these codes to human-readable text, so that we can use it later on for classification. \\

As stated above, we would like to link P2000 notifications to additional data sources.
Twitter is an interesting source of information, as it is a very fast-pace and accessible medium.
As a result, when an incident occurs, corresponding messages often appear within a very small time frame.
Additionally, Twitter offers an API which facilitates data retrieval.

\subsubsection*{Approach}
First of all, we should obtain a list of P2000 notification codes, which may differ per region.
In the database, we could add a table with the notification code as key, and the meaning as an attribute.
Later on, when indexing the P2000 notifications, this table can be referenced to gather the meaning of the codes. \\

As for the Twitter data collection, Twitter offers an API which allows us to easily query Twitter.
We will search for Twitter notifications within a specified time frame and a location, depending on the P2000 notifications we receive.

\subsubsection*{Evaluation }
Since both of these data sources are of a rather strict format, running the program a few times should be sufficient to evaluate the behaviour. 