\subsection{Collect news feed}
\textbf{Name:} Guus van Lankveld \indent \textbf{StudentNumber:} 0629468

\subsubsection*{Motivation}
One of the projects goals was to associate P2000 notifications with relevant local news articles. In order for the system to be able to process these articles they need to be retrieved from the web. When examining a number of news websites (such as \url{http://www.ed.nl} or  \url{http://www.rtvnh.nl}) it appeared that:
\begin{enumerate}
\item There is no universal page design for news articles.
\item There is no straightforward querying method that applies to all news websites (no general API).
\item All news article pages contain links to related article pages.
\item Most news article pages contain a large amount of noise (e.g. advertisements, headers, footers)
\item All news pages use HTML as markup language.
\end{enumerate}

\subsection*{Approach}
When taking notion 2 and 3 (in the list above) into account, it is clear that a web crawler is suited for the retrieval task. This crawler will attempt to index all the news articles within a specified date range for a number of specified news websites. The websites that were chosen for this project are:
\begin{itemize}[noitemsep]
\item RTV Rijnmond (Zuid-Holland): \url{http://www.rijnmond.nl/}
\item RTV Noord-Holland: \url{http://www.rtvnh.nl/}
\item Omroep Gelderland: \url{http://www.omroepgelderland.nl/}
\item Omroep Noord-Brabant: \url{http://www.omroepbrabant.nl/}
\end{itemize}
These websites were chosen because they contain a large number of articles about local news, which is more relevant to the project than for instance national or international news.\\
The crawler will need to operate using the following rules:
\begin{enumerate}
\item Crawling will start at a specified page on the news website (seed).
\item Crawling will be executed using a set of politeness rules: a maximum number of crawls per minute are maintained and robots.txt is respected.
\item Only links that lead to the domain of the specified news website will be crawled, otherwise the probability that irrelevant websites are indexed (for instance: advertisers) is too high.
\item The crawler operates on a single thread. This is sufficient considering the limited scope of the project and reduces the time needed to implement the crawler.
\end{enumerate}
\subsection*{Implementation and evaluation}
The crawling algorithm used for the crawler has the following input:
\begin{itemize}[noitemsep]
\item $url = $ the (seed) url where the crawler will start
\item $nrOfDaysBack = $ the number of days the crawler will 'look back': pages older than $nrOfDaysBack$ days are not processed.
\item $timeToLive = $ how long the algorithm will keep running.
\end{itemize}
The crawling algorithm (in simplified form) executes the following steps:
\begin{enumerate}
\item intitialize $urlForntier = \emptyset$; $alreadyCrawled = \emptyset$
\item Retrieve the page at $url$ and determine its $rootUrl$ (for instance: if $url = $\url{http://www.omroepbrabant.nl/nieuws/8346234} then $rootUrl = $\url{http://www.omroepbrabant.nl}.
\item Process the page by extracting the date and removing all scripts, images and other html tags deemed irrelevant. Then, retrieve all the pages' relevant links $L$, determined by their rootUrl. Put each link $l \in L$ into $urlFrontier$ if $l \not\in alreadyCrawled$. 
\item Store the processed page into the project's database.
\item Put $url$ into $alreadyCrawled$
\item $url := urlFrontier[0]$ and then remove item at index 0 from $urlFrontier$.
\item Update $timeToLive$ by subtracting the amount of time that has passed while executing the algorithm so far.
\item Repeat steps 2-7 until $urlFrontier = \emptyset \vee timeToLive \le 0$
\item Store 
\end{enumerate}
Evaluation of the crawler was done by setting the $timeToLive$ to a few minutes and checking the logs and database manually and determining if the following properties held for all four chosen news websites:
\begin{enumerate}
\item Only pages in the $rootUrl$ domain are retrieved
\item Most of the retrieved pages are news articles
\item The extracted dates matches the one in the articles
\item No relevant HTML tags were removes (the content of the article has remained intact)
\end{enumerate}
Fulfilling these requirements enables further processing of the article (such as extracting the body of the article tokenization and stemming) which is done by other modules of the retrieval system.\\
\subsection*{Improvements and future work}
Because of the scope of this project and the amount of time allocated to it, the crawler has a simple design. A few improvements which could be implemented in future projects:
\begin{itemize}
\item The current implementation uses  a single thread. If multiple threads would be used, then more pages could be crawled at the same time.
\item The crawler assumes that every news-article it has already crawled, does not change anymore. Therefore, known urls are not crawled again. This was sufficient for our project and saves time that would otherwise be spent on crawling (unchanged) pages which were already indexed. In future implementations, a more sophisticated method of checking  crawled pages could be used.
\end{itemize}