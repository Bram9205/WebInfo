\subsection{Cluster notifications}
\textbf{Name:} Bram Kohl \indent \textbf{StudentNumber:} 0746107

\subsubsection*{Motivation}
When an incident that requires emergency services happens, there are often multiple notifications in the P2000 system concerning that single incident. Take for example a severe car accident. This often requires the fire department to open cars, the police department to organize the other traffic and one or more ambulances to take care of the passengers of the cars involved in the accident. These all show up as separate notifications in the P2000 system. As we want to be able to associate the P2000 notifications with news messages and twitter messages, we would like to combine these notifications that belong to the same incident. This way, the association analyzer has more data to work with, resulting in more reliable associations.
\subsubsection*{Approach}
As input, we have many P2000 notifications and as output we would like to know which P2000 notifications belong together. There is no hierarchy in our clustering, we just have non-overlapping clusters, so we apply our own partitional clustering algorithm.\\

We cluster notifications when they occur in the same region and on the same postal code, within a time frame of 30 minutes. It is possible to chain notifications creating a cluster of which the first notification was received more than 30 minutes before the last. For example if we have a notification A at 12:00, notification B at 12:20 and notification C at 12:40 on the same day, they would all be clustered together if they have the same postal code and region.\\

We start by retrieving all notifications in chronological order. Then for each notification we check whether there is a notification with the same postal code and region, which was received at most 30 minutes ago (of course, for the first notification there are none). As we handle the notifications in chronological order, the found notifications (which were received in the past relative to the current notification) have already been handled. If we find any notifications that should be clustered, we set the cluster of the currently being handled notification to the cluster of the notification with the highest id of the found notifications. This is just for tie breaking though, it does not really matter which one we pick, they should all be clustered together already and thus have the same cluster id. If we do not find any notifications that should be clustered, we set the cluster to the own id.\\

After running this algorithm, every notification belongs to a cluster. Some just to their own cluster (cluster of 1) and others together in a cluster.
\subsubsection*{Evaluation}
The clustering of notifications has been evaluated manually, by searching for a few P2000 notifications that should be matched together and checking whether they are properly indicated as belonging together. Furthermore, we checked whether messages that should not be matched are not indicated as belonging together as well. This results in about 5000 clusters, most of size 2. In our evaluation test set we found most clusters and did not get any false positives (e.g. clustered notifications that should not be clustered). The clustering took about one hour to run for about 35000 notifications.
