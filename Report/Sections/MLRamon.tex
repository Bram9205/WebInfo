\subsection{Associate the data sources}
\textbf{Name:} Ramon de Vaan \indent \textbf{StudentNumber:} 0758873

\subsubsection*{Motivation}
As stated in the introduction, one of the main goals for the project is to find news sources that are related to a certain P2000 notification.
From P2000 notifications alone, it is difficult to fully grasp the events that occurred at a certain location, as they tend to be rather basic, containing only basic information.
If incidents occurs, the first slivers of information regarding these are provided by local news media and twitter sources, often within a very small time frame. \\
Thus, to provide the user with as much information as possible, we would like to link these information sources to the lacking P2000 notifications.

\subsubsection*{Approach}
At this step in the process, we will have already mined keywords from both P2000 notifications and local news/twitter sources, with hopes these accurately describe what the corresponding source is about.
We will need to develop an algorithm that can produce a measure of relevance between a pair of these keyword sets.
There might be different options and/or configuration for the algorithm we will use, but most likely several will need to be tested and compared. \\

Given a certain location and timespan, we will find the corresponding P2000 notifications.
For each of these, we will be iterating over a (sub)set of data sources, producing a relevance score between each P2000 - data source pair.
We will then sort the data sources based on the relevance score our algorithm produces.
The top-most relevant sources can then be displayed to the user.
Additionally, it might be wise to declare a lower bound for the relevance score. 
It might be better to show the user there were no relevant data sources than to have the user read those that are not relevant at all. \\

There are a few additions to this approach, however.
We will need to filter the results depending on the options the user has selected.
For example, the user may have selected to only show twitter messages.
If possible, it would be best to apply a filter on the data sources before computing the relevance, as the latter is probably a more costly operation, so running that on fewer entries would increase performance. \\
Additionally, it might be wise to not only show the top-most relevant sources, but try to maintain a certain variance in data sources.
For example, it might be better to show a few local news articles instead of solely twitter messages, even if they have a higher relevance score.



\subsubsection*{Evaluation}
There is no solid means to evaluate the performance of the algorithm apart from user testing.
As stated above, we will have to test different algorithms and manually compare the results.