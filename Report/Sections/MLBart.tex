\subsection{Classification of P2000 notifications }
\textbf{Name:} Bart van Wezel \indent \textbf{StudentNumber:} 0740608

\subsubsection*{Motivation}
Not all P2000 notification are really clear. A example of an notification can be: \\ 
10-09-15	16:05:06	Ambulance	Midden- en West Brabant	B1 5011ME X : Mahlerstraat X Tilburg 67893  \\
This notification is not really interesting to see, because when you get to see such a notification you probably have no idea what happened. 
With a bit of research you can find that B1 means ordered transport. For example from the hospital to a nursing home. 
We are going to try to classify the P2000 notifications, such that we can get a clear classification for each P2000 notification.
This way we can make the program more useful for a general public. 

\subsubsection*{Approach}
To classify the P2000 notifications, we will try to find a main classification and a sub-classification. 
A main classification could be some issue with the train. 
The sub classification could be fire on the track. 
To find those classifications we have to write a algorithm that tries to classifies the notifications. 
Those classification can be done with a list of all abbreviations and use this list to determine which main classifications and sub classifications there should be. 

\subsubsection*{Evaluation }
We can evaluate this part by manually classifying  multiple P2000 notifications and then apply the algorithm to those P2000 notifications.
When the result of the algorithm is the same as the manually classified notifications, we know that the algorithm probably works correctly.