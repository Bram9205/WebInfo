\section{Project Conclusion}
In the end, we managed to create a runnable program that takes a city name as input, and produces a set of pins on a map of the Netherlands.
These pins are clickable, and doing so will provide the user with a set of relevant news articles and twitter messages.
As such, the user is able to inspect a given neighbourhood, even within a given time period, much as we planned.
In doing so, we have applied many of the techniques discussed in class.
In that respect, we have succeeded in our project. \\
However, the project does have some shortcomings.
First and foremost, in its current state, a lot of the data is provided manually in advance, instead of being constantly and automatically updated.
Additionally, partly because of the short amount of time we had and the reason given above, the amount of data we have gathered is limited.