\subsection{Classify news feed / Twitter data}
\textbf{Name:} Jasper Selman \indent \textbf{StudentNumber:} 0741516

\subsubsection*{Motivation}
People often want to know what is happening in the neighbourhood, but they do not want to know every little thing that is happening.
They are only interested in certain type of news. This is why we want to classify the news feed / Twitter data such that people can choose from a set of subjects. For example people only want news about events which involved an ambulance. 

\subsubsection*{Approach}
The real approach to do this is yet to be researched and determined, but an initial idea is to first create a set of subjects we offer the user to choose from. For every possibility in that set we should create a set of relevant similar words. Next to that we should classify every news item with some kind of algorithm such that if one of the words of the set for a search term occurs enough (the amount has to be determined) in the news item, then the item is classified for that search term.

\subsubsection*{Evaluation }
This will be checked manually, because it is very easy to check, when you choose a subject, if the resulting news feed shown indeed involved the subject or that also other news is shown.
